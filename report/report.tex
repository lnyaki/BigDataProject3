\documentclass[a4paper,11pt]{article}
\usepackage[utf8]{inputenc}
\usepackage{textcomp}
\usepackage{lmodern}
\usepackage{listings}
\usepackage{graphicx}
\usepackage{listings}
\usepackage{color}
\definecolor{lightgray}{rgb}{0.9,0.9,0.9}
\definecolor{darkgray}{rgb}{0.4,0.4,0.4}
\definecolor{purple}{rgb}{0.65, 0.12, 0.82}
\usepackage{url}
\usepackage[top=3cm,bottom=3cm,left=3cm,right=3cm]{geometry}

\title{UNamur\\
	ICYBM201 Big Data and Computer Security : Fame for sale, efficient detection of fake Twitter
followers}

\author{TIO NOGUERAS Gérard, NYAKI Loïc}

\begin{document}
\maketitle

\newpage
\tableofcontents
\newpage

\section{Introduction}

\section{Data extraction}
\subsection{Available data}

\subsection{Table 1 creation}
In this section we are going to create the base dataset that we will use throughout the project.
We have 5 available datasets: 
We are going to create 1 final dataset. The BAS dataset constituted of 1950 human twitter accounts and 1950 fake accounts.\\ The human accounts are simply the sum of the human datasets we had available.\\
For the fake accounts, we randomly undersampled the 3 datasets available to obtain the same number of accounts as the normal ones.\\
After undersampling the users, we used the ids of these users to collect the rest of the data in the other files.
\bibliography{bibliography}
\bibliographystyle{plain}
\end{document}
